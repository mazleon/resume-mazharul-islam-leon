%%%%%%%%%%%%%%%%%%%%%%%%%%%%%%%%%%%%%%%%%
% Medium Length Professional CV
% LaTeX Template
% Version 2.0 (8/5/13)
%
% This template has been downloaded from:
% http://www.LaTeXTemplates.com
%
% Original author:
% Trey Hunner (http://www.treyhunner.com/)
%
% Important note:
% This template requires the resume.cls file to be in the same directory as the
% .tex file. The resume.cls file provides the resume style used for structuring the
% document.
%
%%%%%%%%%%%%%%%%%%%%%%%%%%%%%%%%%%%%%%%%%

%----------------------------------------------------------------------------------------
%	PACKAGES AND OTHER DOCUMENT CONFIGURATIONS
%----------------------------------------------------------------------------------------

\documentclass{resume} % Use the custom resume.cls style
\usepackage{hyperref}
\usepackage[left=0.75in,top=0.5in,right=0.75in,bottom=0.5in]{geometry} % Document margins
\newcommand{\tab}[1]{\hspace{.2667\textwidth}\rlap{#1}}
\newcommand{\itab}[1]{\hspace{0em}\rlap{#1}}
\name{Mazharul Islam Leon} % Your name
\address{Dhaka, Bangladesh} 
%\address{123 Pleasant Lane \\ City, State 12345} % Your secondary addess (optional)
\address{(+880)1911687821 \\ mzleon.cse@gmail.com} % Your phone number and email

\begin{document}

%----------------------------------------------------------------------------------------
%	EDUCATION SECTION
%----------------------------------------------------------------------------------------

\begin{rSection}{Education}

{\bf United International University, Dhaka} \hfill {\em Oct 2015 - Oct 2020} 
\\  B.Sc. in Computer Science and Engineering \hfill {CGPA: 3.03}
%Department of Chemical Engineering  
%Minor in Linguistics \smallskip \\
%Member of Eta Kappa Nu \\
%Member of Upsilon Pi Epsilon \\


\end{rSection}

%----------------------------------------------------------------------------------------
%	Career Objective
%-----------------------------------------------


% \begin{rSection}{Career Objective}
%  To work for an organization which provides me the opportunity to improve my research skills and knowledge to grow along with the organization objective.
% \end{rSection}

%----------------------------------------------------------------------------------------

%----------------------------------------------------------------------------------------
%	TECHNICAL STRENGTHS SECTION
%----------------------------------------------------------------------------------------

\begin{rSection}{Technical Strengths}
{\small
\begin{tabular}{ @{} >{\bfseries}l @{\hspace{4ex}} p{0.62\textwidth} }
Languages & Python (Advanced), C++, Java, SQL, Bash Scripting \\
Generative AI \& LLMs & LangChain, LangGraph, Google ADK, RAG Pipelines, Autonomous Agents, LLM Fine-tuning (LoRA, QLoRA, PEFT), LLM Evaluation (BLEU, ROUGE), Hugging Face \\
Deep Learning \& CV & PyTorch, TensorFlow, CNNs, Vision Transformers (ViT), Object Detection (YOLO v12), OpenCV \\
ML System Design & Microservices Architecture, Scalable Systems, RESTful APIs, Real-time Inference, Event-Driven Architecture \\
MLOps \& Cloud & AWS (SageMaker), Docker, Github Actions, CI/CD, MLflow, Airflow, Model Deployment \& Monitoring \\
Data \& Storage & PostgreSQL, MySQL, Vector Databases (Chroma, Qdrant), Data Pipelines (ETL) \\
Edge AI & NVIDIA Jetson, DeepStream, TensorRT, Edge Computing \\
\end{tabular}
}

\end{rSection}

%----------------------------------------------------------------------------------------
%	WORK EXPERIENCE SECTION
%----------------------------------------------------------------------------------------

\begin{rSection}{Work Experience}
\begin{rSubsection}{RedDot Digital Ltd. - a subsidiary of Robi Axiata Limited}{August 2021 - January 2026}{Senior Software Engineer (Team: IoT \& ML)}{Full-time}

\item Architected and deployed a computer vision pipeline for automated tobacco leaf grading, integrating image classification, segmentation, and MLflow tracking — boosted grading accuracy by 25\% and reduced manual processing time by 3x.
\item Designed and implemented a hybrid YOLO + ML model for real-time video anomaly detection on NVIDIA Jetson edge devices — achieving \textless 50ms inference latency at 30 FPS in production environments.
\item Delivered a high-throughput facial recognition system achieving 50+ TPS on CPU with 99.2\% verification accuracy, serving 30M+ telecom users with secure biometric verification at national scale.
\item Built and optimized multiple LLM-powered chatbot systems using LangChain, Google ADK, and vector DBs (Chroma, Qdrant) — reduced customer support response time by 60\% and handled 10K+ daily queries autonomously.
\item Engineered real-time defect detection for manufacturing using classical CV + YOLO models — improved detection precision to 96.5\%, reducing product wastage by 30\%.
\item Designed and deployed an OCR + face matching eKYC platform supporting 20M+ fintech users — reduced onboarding time from 48 hours to under 5 minutes while maintaining regulatory compliance.
\item Spearheaded 5+ Proof of Concept (PoC) projects, successfully transitioning 3 into revenue-generating products, demonstrating strong project management and business acumen.

 

\end{rSubsection}


\begin{rSubsection}{CMED Health Ltd., Bangladesh}{Sep 2020 - July 2021}{Junior Data Science Engineer}{Full-time}
 \item Built data extraction and ETL pipelines from large SQL dumps for analytics and ML use cases.
 \item Performed data cleaning, preprocessing, and exploratory analysis to support research and business insights.
 \item Collaborated with cross-functional teams to translate analytical findings into actionable decisions.
 \end{rSubsection}


\end{rSection}

\newpage

%	Projects
%----------------------------------------------

\begin{rSection}{Projects}
{\bf AI-Powered Biometric Verification System – Robi Axiata PLC}
\\ Developed an AI-driven biometric verification system with face recognition and camera-based fingerprint matching for SIM registration. Expected to be used by 30 million users with 50+ TPS rate.

{\bf AI Based RedDot eKYC for Telecash} 
\\ Developed an AI-powered eKYC system for Telecash, Southeast Bank PLC's mobile banking app, featuring face identification, face matching, and OCR-based NID extraction. Expected to onboard 20 million users.

% {\bf Facemask detection using deep learning model}
% \\While doing Machine Learning course I developed this project.This application can detect face-masks using a web camera. The deep learning model was trained using the Fastai libraries with the ResNet-50 model. The dataset is scraped from the internet and for deployment, it uses OpenCV
% libraries.[GitHub: \url{https://github.com/princexoleo/face-mask-detector}]

\end{rSection}





%	Achievements
%----------------------------------------------------------------------------------------

\begin{rSection}{Achievements}
\begin{rSubsection}{}{}{}{}
\item Received the prestigious Star Developer Award for  Q1-2022 \& Q3-2023 at RedDot Digital Ltd.
\item Achieved the runner-up position in the SDG Hackathon 2.0 organized by Banglalink.
\item Received academic scholarships ranging from 25\% to 50\% for multiple trimesters.
 
 \end{rSubsection}
 \end{rSection}


 \begin{rSection}{Extra Activities}
 \begin{rSubsection}{}{}{}{}
\item Instructor for the ``Build with AI'' event organized by the Google Developer Group (GDG) at UIU.
\item Instructor for the ``Intro to Computer Vision'' event organized by the Quantum AI.
\item Vice President, UIU App Forum, from Jan 2020 to Sep 2020.
\item Member, Competitive Programming (CP), at UIU
\end{rSubsection}
 \end{rSection}



%--------------------------------------------------
%--------------------Research-----------------
%-------------------------------------------------


\begin{rSection}{Research Profile}
\item Publication [1]: {\bf Mazharul Islam Leon}, Md Ifraham Iqbal, Sayed Mehedi Azim, and Khondaker A.
Mamun ``Predicting COVID-19 infections and deaths in Bangladesh using Machine \\ Learning Algorithms''- ICICT4SD 2021 - IEEE-Xplore. (Status: Published)

\item Publication [2]: Afnan Islam, Thajid Ibna Rouf Uday, Nazib Ahmad, Md Toriqul Islam, Amit
Ghosh, Sadia Kamal, Tanzir Ahommed, {\bf Mazharul Islam Leon}, Ehsan Ahmed
Dhrubo, ``EduBot: An Educational Robot for Underprivileged Children''- ICACTM 2019 - IEEE-Xplore. (Status: Published)

\item Publication [3]: {\bf Mazharul Islam Leon}, Md Ifraham Iqbal, Sadaf Meem, Furkhan Alahi, Morshed
Ahmed, and Md. Saddam Hossain Mukta ``Dengue Outbreak
Prediction from the Weather Spatio-temporal data using Deep Learning''
-ICBBDB 2021, Springer. (Status: Published)


\item Publication [4]: Md Ifraham Iqbal, {\bf Mazharul Islam Leon}, Jahidul Islam Rahat, Nilamber Haider
Tonmoy, and Amit Ghosh ``Deep Learning Based Smart Parking Management
System For Metropolitan City'' - Tensymp 2021 - IEEE-Xplore. (Status: Published)

\item Publication [5]: Khondaker A. Mamun, Moinul H. Chowdhury, Rubaiyat Alam Hridhe, Tanvir Islam,
{\bf Mazharul Islam Leon}, Mithila Faruque, Mohammad Badruddozza Mia, Md Jasim
Uddin and Farhana Sarker ``Implementation of a Digital Healthcare Service
Model for Ensuring Preventive and Primary Healthcare in Rural Bangladesh'' - IC4IR 2021, Springer. (Status: Published)

\item Publication [6]: Syed Ahmed, 
{\bf Mazharul Islam Leon}, Sanchita Pal, M. Rubaiyat Hossain Mondal ``Hybrid CNN-LSTM Transfer Learning for Dengue Diagnosis from Raman Spectroscopy Images'' - ICTP 2023, IEEE Xplore. (Status: Published)

\end{rSection}

%--------------------------------------------------
%------------------Extra----------------------
%---------------------------------------------
\newpage
 
 
% \begin{rSection}{Extra-Cirrucular} 
% \begin{rSubsection}{}{}{}{}
% \item Vice President, UIU APP FORUM (2020-2021)
% \item Executive Member of the  UIU Robotics Club since 2017.
% \item Member of the  UIU Computer Club since 2017.
% \item Participate in a competitive programming contest (CP) and session at UIU.
% \item Take a workshop on Programming Language C hosted by the UIU App Forum.
% \item Take a workshop on Arduino hosted by the UIU Robotics Club.
% \item Take a workshop on Machine Learning \& Computer Vision organised by the Quantum AI.
%  \end{rSubsection}

% \end{rSection}

% %---------------------------------------------
% %--------------------Reference-----------
% %--------------------------------------------------



% %----------------------------------------------------------------------------------------
% \begin{rSection}{References}
% \vspace{0.3cm}
% \itab{\textbf{Prof. Dr. Dewan Md. Farid}} \tab{}  \tab{\textbf{Dr. Swakkhar Shatabda}}
% \\ \itab{Professor of Computer Science } \tab{}  
% \tab{Professor \& Director - IQAC}
% \\ \itab{United International University} \tab{}  \tab{United International University} 
% \\ \itab{Email: dewanfarid@cse.uiu.ac.bd} \tab{}  
% \tab{Email: swakkhar@cse.uiu.ac.bd} 
% \\ 
% \\ 
% \itab{\textbf{Md. Mofijul Islam}} \tab{}  
% \tab{\textbf{}}
% \\ \itab{PhD Student at Link Lab } \tab{}  
% \tab{}
% \\ \itab{University of Virginia} \tab{}  \tab{} 
% \\ \itab{Email: mi8uu@virginia.edu} \tab{}  
% \tab{}
% \\ 
% \itab{Websites: https://mmiakashs.github.io}
% % \\ \itab{Process Control (ongoing)} \tab{} \tab{Electrodynamics}

% \end{rSection}



\end{document}
